\problem*{Let $f:\mathbb{N}\to\mathbb{N}$ be the $n$-th digit in the decimal
representation of $\pi$. Prove that $f\in\GRF$.}

\begin{pf} \rm
 Given a $m$ by $m$ grid, we count the integral point of $(x,y)$
 within a circle centered at $(0, 0)$ with radius $m$ by 
\[
 S = \Big| \{ (x, y) ~|~ x,y\in\mathbb{N}\textrm{ and }x^2 + y^2 \leq m^2 \} \Big|
\]

\noindent to approximately find $\pi$. $S$ is elementary because
\[
 S(m) = \sum_{i=0}^{m} \sum_{j=0}^{m} N(i^2 + j^2 - m^2).
\]

\noindent Area of the circle is $S_c(m) = \pi m^2 / 4$, 
and by the fact that the circle intersects with at most $2m$ $1\times 1$ blocks,
we have $|S(m)-S_c(m)|<2m$, therefore

\[
\begin{array}{c}
 \displaystyle
   \left| 
   \frac{S(m)}{m^2} - \frac{\pi}{4}
   \right| = \frac{1}{m^2} |S(m) - S_c(m)| < 2 m^{-1}, \textrm{ and}\\
 \displaystyle
 \big| 4S(m) - m^2 \pi\big| < 8m.
\end{array}
\]

To compute $f(n)$, we need an exponentially large grid, say, $m=10^{n+1}$.
Then we have $|4S(10^{n+1}) - 10^{2n+2}\cdot\pi| < 10^{n+2}$,
which means that $4S$ has at least 
$n+1$ accurate digits of $\pi$, thus
$f(n) = \rs( S(10^{n+1}) / 10^{n+2} , 10) \in\EF\subset\GRF$. 
\qed
\end{pf}

